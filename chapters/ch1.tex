\documentclass[float=false, crop=false, class=report, a4, 12pt, onesided]{standalone}
\usepackage{standalone}

\usepackage{mthesis-math-chapters}

\begin{document}

\ifstandalone
	\setcounter{chapter}{0}
\fi

\chapter{Getting started}

This template is intended for use by students in the School of Mathematics who
are starting to write their thesis. This template is designed to take advantage
of the \LaTeX packages that can be used to build a beautiful monolithic project
in a way that is maintainable and modular.

This template has the following folder structure.

\begin{figure}[ht]
	\centering
	\scalebox{0.65}{
		\subimport{../img/}{dtree}
	}
	\caption{The directory structure of the thesis template.\label{fig:dirstruct}}
\end{figure}

This folder structure is designed to be as straight forward as possible. Each
chapter is written in an independent and compilable file, each tikz image can
be written in the chapter in which it appears or can be written in the image
folder where it can be imported for improved modularity.

This modularity is achieved through the use of two packages, the standalone
package and the import package, discussed in \Cref{cha:standalone}. As most
mathematics students will also be required to battle with tikz, we have
included the `./tikz/' folder and the tikz library package `external' described
in \Cref{cha:tikz}.

Lastly, the aim of this project is to allow for quick and easy modification of
the preamble in a way that is minimally disruptive. This detail is described in
\Cref{cha:preamble}. For examples related to the preamble as it is given, see
\Cref{app:usages}.

\ifstandalone
	\bibliography{../thesis-refs}
\fi

\end{document}
